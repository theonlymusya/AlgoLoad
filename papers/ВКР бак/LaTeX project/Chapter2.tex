\section{Обзор существующих решений}
\label{sec:Chapter2} \index{Chapter2}
% \todo[inline]{Здесь надо рассмотреть все существующие решения поставленной задачи, но не просто пересказать, в чем там дело, а оценить степень их соответствия тем ограничениям, которые были сформулированы в постановке задачи.}

\subsection{Развитие возможностей трехмерной визуализации графов и ее применение в науке}

Визуализация графов прошла долгий путь с первых простых двумерных представлений. Были достигнуты значительные успехи в возможностях и методах, используемых для представления сложных данных в трехмерном виде. Эти разработки были вызваны необходимостью визуализации и анализа данных из различных областей, включая науку и бизнес. В настоящее время Проводится много исследований в области визуализаций графов. \cite{GraphDrawingAlgorithms} \cite{AutomaticGraphDrawingAlgorithms} \cite{TechniqueForDrawingDirectedGraphs} \cite{MethodsAndToolsForVisualizationOfGraphsAndGraphAlgorithms}

Одной из областей науки, в которой требуются передовые методы представления трехмерных графов, является биоинформатика. С появлением генетических данных исследователям понадобились способы визуализации и анализа взаимодействий белков, генов и других молекулярных структур. Использование 3D-визуализации позволило ученым лучше понять взаимосвязь между этими структурами и определить потенциальные цели для разработки лекарств.

Помимо науки, бизнес и финансы также требуют передовых методов визуализации графиков. С появлением больших данных компаниям понадобились способы лучше понять сложные взаимосвязи между различными точками данных. Трехмерная визуализация графов данных позволила более интуитивно представить эти отношения, что позволяет принимать более обоснованные и взвешенные решения.

Эволюция возможностей визуализации графиков была обусловлена необходимостью лучше понимать сложные данные из различных областей. Использование 3D визуализации позволило получить более интуитивное и всестороннее понимание этих взаимосвязей, что привело к прорывам в исследованиях и принятии решений. \cite{VisualAnalysisOfGraphAlgorithmDynamics}

Анализ производительности алгоритмов с использованием информационных графов не является достаточно популярной областью в науке и не содержит устоявшиеся методов по анализу таких сложных и многомерных данных. Поэтому для понимания способов реализации проекта требуется дополнительное рассмотрение популярных технологий трехмерной визуализации, доступных на сегодняшний день.









\subsection{Популярные методы 3D-визуализации}

Трехмерная визуализация — важный аспект взаимодействия  с пользователем при разработке программного обеспечения. Рассмотрим десять самых популярных методов 3D-визуализации, существующие на сегодняшний день.

\subsubsection{OpenGL}

OpenGL — это широко используемый графический API, который часто используется для 3D-визуализации \cite{OpenGL_offc}. Это программная библиотека с открытым исходным кодом, предоставляющая набор функций для создания 2D- и 3D-графики. OpenGL — это кроссплатформенный API, что означает, что его можно использовать на различных платформах, включая Windows, Linux и macOS. Он имеет долгую историю и широко поддерживается производителями оборудования, что делает его популярным выбором для 3D-визуализации. 

\subsubsection{WebGL}

WebGL — это API JavaScript, который позволяет разработчикам создавать трехмерную графику в веб-браузерах \cite{WebGL_offc}. Он предоставляет набор низкоуровневых API для создания высокопроизводительной 3D-графики в Интернете, включая поддержку шейдеров, текстур и геометрии. WebGL поддерживается всеми основными веб-браузерами и используется в различных приложениях, включая научную визуализацию.  \cite{WebGLGraphicsandAnimation} \cite{ResearchandApplicationofWeb3D}

\subsubsection{Three.js}

Three.js — это популярная библиотека JavaScript, основанная на WebGL, которая используется для создания 3D-графики в веб-приложениях \cite{Threejs_offc} . Он предоставляет набор инструментов и функций для создания интерактивных 3D-анимаций и визуализаций, включая график сцены, элементы управления освещением и камерой, а также поддержку нескольких форматов файлов. Three.js прост в использовании и имеет большое сообщество разработчиков, которые вносят свой вклад в библиотеку и делятся примерами и учебными пособиями. \cite{ThreejsFramework}

\subsubsection{Babylon.js}

Babylon.js — еще одна популярная библиотека JavaScript, которая используется для создания 3D-графики в веб-приложениях \cite{Babylonjs_offc}. Она предоставляет набор инструментов и функций для создания высококачественной 3D-анимации и визуализации, включая поддержку физического моделирования, систем частиц и эффектов постобработки. Babylon.js так же прост в использовании и имеет большое сообщество разработчиков. 

\subsubsection{VTK}

VTK (Visualization Toolkit) — это программная система с открытым исходным кодом, которая используется для создания 3D-графики в медицинской визуализации, научной визуализации и инженерных приложениях \cite{VTK_offc}. Он предоставляет набор инструментов и функций для создания высококачественных 3D-визуализаций, включая поддержку объемного рендеринга, изоповерхностей и интерактивного исследования. VTK широко используется в научных исследованиях и промышленности.

\subsubsection{OpenSceneGraph}

OpenSceneGraph — это мощный набор инструментов для трехмерной графики с открытым исходным кодом, который используется для создания высокопроизводительных трехмерных визуализаций \cite{OpenSceneGraph_offc}. Он предоставляет набор инструментов и функций для создания сложных 3D-сцен и анимации, включая поддержку обхода графа сцены, оптимизацию геометрии и программирование шейдеров. OpenSceneGraph так же широко используется в исследованиях и промышленности. 

\subsubsection{DirectX}

DirectX — это набор API-интерфейсов, которые используются для разработки мультимедийных приложений на платформах Microsoft \cite{DirectX_offc}. Он предоставляет набор инструментов и функций для создания 3D-графики и анимации, включая поддержку шейдеров, текстур и устройств ввода. DirectX широко используется в индустрии разработки игр. 

\subsubsection{Unity}

Unity — популярный игровой движок, который используется для разработки 2D- и 3D-игр \cite{Unity_offc}. Он предоставляет ряд инструментов и функций для создания высококачественной графики и анимации, включая мощный редактор и язык сценариев. 

\subsubsection{Blender}

Blender — это программное обеспечение для 3D моделирования и анимации с открытым исходным кодом, которое используется для создания высококачественной 3D-графики и анимации \cite{Blender_offc}. Он предоставляет широкий спектр инструментов и функций для создания сложных 3D-сцен и анимации, включая поддержку моделирования, текстурирования, оснастки и рендеринга. 

\subsubsection{Maya}

Maya --- это программное обеспечение для 3D-моделирования и анимации, которое широко используется в индустрии кино и видеоигр \cite{Maya_offc}. Он предоставляет набор инструментов и функций для создания высококачественной 3D-графики и анимации, включая поддержку моделирования, текстурирования, оснастки и рендеринга. 