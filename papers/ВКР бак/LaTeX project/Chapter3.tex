\section{Исследование и построение решения задачи}
\label{sec:Chapter3} \index{Chapter3}
% \todo[inline]{Здесь надо декомпозировать большую задачу из постановки на подзадачи и продолжать этот процесс, пока подзадачи не станут достаточно простыми, чтобы их можно было бы решить напрямую (например, поставив какой-то эксперимент или доказав теорему) или найти готовое решение.}


\subsection{Выбор языка программирования}

От системы требуется чтобы ей мог воспользоваться каждый желающий, без установки дополнительного программного обеспечения и знания глубоких теоретических сведений о методах построения информационных графов алгоритмов. Для обеспечения этих требований система должна обладать следующими качествами:

\begin{itemize}
    \item Гибкость настройки и простота запуска.
    \item Кроссплатформенность.
    \item Независимость от дополнительного ПО.
\end{itemize}

Этими качествами обладают системы, развернутые при помощи инструментов браузера с использованием распространенных стандартных библиотек, которые есть на каждом компьютере изначально. То есть для обеспечения максимальной независимости и доступности исследуемая система может быть написана на некотором серверном языке и кроссбраузерной языке, гарантированно доступным на клиентской стороне. Выбор инструментов разработки серверной части системы выходит за рамки данного исследования. Под критерии языка для реализации общедоступной клиентской части попадает только JavaScript. JavaScript обычно используется как встраиваемый язык для программного доступа к объектам приложений. Наиболее широкое применение находит в браузерах как язык сценариев для придания интерактивности веб-страницам. \cite{JavaScript_book}

\subsection{Выбор технологий 3D визуализации}

Проанализировав рассмотренные популярные технологии 3D визуализации, базированные на языке JavaScript, была выбрана библиотека Three.js, которая используется для создания и отображения анимированной компьютерной 3D графики. Данная библиотека была выбрана за счет своей легковесности и кроссбраузерности. 

Для контроля над сценой был использован модуль OrbitControls. Данный модуль является самым распространенным и поддерживаемым решением для реализации управления при помощи компьютерной мышки и клавиатуры \cite{OrbitControls_offc}.  Для создания пользовательского интерфейса и меню управления с возможностями дополнительного анализа графов алгоритмов был использован легковесный модуль dat.GUI \cite{dat_gui_github}. Размер библиотеки со всеми дополнительными модулями пользовательского интерфейса не превышает 1 мегабайта.

\subsection{Выбор архитектуры для разработки - MVC}

Одной из главных задач для достижения цели работы является выбор и разработка подходящей архитектуры приложения. От выбранной архитектуры зависит способ обработки данных и общения с пользователем. Выбор неподходящей структуры и методов общения модулей приложения между собой может значительно повысить нагрузку на вычислительные ресурсы системы и ядро браузера. Рассмотрим основные составные части проекта для решения этой проблемы:

\begin{itemize}
    \item Работа с данными, считывание и создание внутренней структуры графа.
    \item Работа со структурой графа, основываясь параметрах вида, задаваемых пользователем. Создание 3D моделей для каждого объекта в структуре графа.
    \item Создание 3D сцены, содержащей модель графа.
    \item Работа с пользователем: обеспечение связи панели управления с моделью графа, сценой и параметрам и вида.
    \item При изменении параметров вида данные в модели должны локально изменяться и последовательно обновлять структуру графа и 3D модели, представляющие этот граф на сцене.
\end{itemize}

Описанные требования и основные функциональные части являются достаточными критериями для выбора MVC архитектуры. \cite{MVCArchitecture}

\subsubsection{Описание использованной MVC архитектуры}

MVC расшифровывается как Model-View-Controller и представляет собой архитектурный шаблон программного обеспечения, который разделяет данные приложения, пользовательский интерфейс и логику управления на три взаимосвязанных компонента. 

Модель (Model) представляет данные и бизнес-логику приложения, Представление (View) представляет уровень представления приложения, а Контроллер (Controller) действует как посредник между Моделью и Представлением, обрабатывая пользовательский ввод и обновляя Модель и Представление по мере необходимости.

Когда пользователь взаимодействует с приложением, контроллер получает входные данные и решает, как соответствующим образом обновить модель и представление. Модель отвечает за обработку данных и самообновление, а представление отображает данные пользователю в презентабельном формате.

Эта архитектура помогает разбивать сложные приложения на более мелкие, более управляемые компоненты, упрощая разработку, тестирование и обслуживание приложения \cite{DesigningMVCModelforRapidWebApplicationDevelopment}. Это также упрощает сотрудничество между разработчиками, работающими над разными частями приложения.


\subsubsection{Как MVC помогает решать проблемы с разделением ответственности между модулями приложения}

Одной из классических проблем при разработке приложений является разделение ответственности между обработкой данных и пользовательским интерфейсом. В архитектуре MVC эта проблема решается путем разделения модели и представления на отдельные компоненты, каждый из которых несет свою ответственность.

Другой распространенной проблемой является сложность внесения изменений в приложение без нарушения других частей кода. В архитектуре MVC изменения можно вносить в один компонент, не затрагивая другие, при условии, что интерфейсы между компонентами остаются прежними.

MVC помогает повысить общую производительность приложения за счет уменьшения дублирования кода и улучшения организации кода. Это упрощает выявление и исправление ошибок, а также делает приложение более масштабируемым и адаптируемым к изменяющимся требованиям. В целом, архитектура MVC — это мощный инструмент для решения многих классических проблем разработки приложений, и сегодня она широко используется в отрасли разработки.


\subsection{Разбиение ответственности между модулями приложения}

Model-View-Controller. Она состоит из модели (Model) со структурой графа, части отображения (View) с набором методов преобразования данных в 3D модели и контроллера (Controller), содержащего в себе методы изменения модели и обновления визуализации через пользовательский интерфейс.

Выбрав архитектуру, требуется грамотно распределить задачи и ответственность между модулями приложения, следуя правилам конкретной архитектуры. Рассмотрим, как в данном случае распределена ответственность.

\subsubsection{Содержание модуля Model}

\begin{itemize}
    \item Обработка входных данных.
    \item Хранение внутренней структуры графа.
    \item Обновление данных в структуре графа по командам пользователя.
\end{itemize}



\subsubsection{Содержание модуля View}

\begin{itemize}
    \item Обеспечение набором методов преобразования данных в 3D модели.
    \item Заполнение сцены 3D моделями, полученными из объектов графа, хранящимися в Model.
\end{itemize}



\subsubsection{Содержание модуля Controller}

\begin{itemize}
    \item Создание графического пользовательского интерфейса. 
    \item Обеспечение связи между пользователем, его командами через меню управления и связкой Model-View.
    \item Управление Model и View через их встроенные методы контроля.
\end{itemize}


