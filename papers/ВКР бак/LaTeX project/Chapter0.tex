\section{Введение}
\label{sec:Chapter0} \index{Chapter0}

% \todo[inline]{В этой части надо описать предметную область, задачу из которой вы будете решать, объяснить её актуальность (почему надо что-то делать сейчас?).
% Здесь же стоит ввести определения понятий, которые вам понадобятся в постановке задачи.}

\subsection{Цель работы} 

Целью данной работы является исследование современных методов 3D визуализации и представление разработки автоматизированной системы визуализации графов алгоритмов. Описываемая система, AlgoView, позволяет составить информационный граф алгоритма, создает его интерактивную 3D модель и предоставляет возможности по его анализу. Такой функционал обеспечивает активное применение системы в образовательных и исследовательских целях. \cite{AlgoWiki_main}

\subsection{Что такое граф алгоритма}
 
Граф алгоритма — это ациклический граф, который представляет операции алгоритма и связи данных между ними. Вершины графа соответствуют основным операциям алгоритма, а дуги — информационным связям между этими операциями. Если две вершины связываются дугой, то вторая вершина использует данные, вычисленные в первой. Граф алгоритма может быть представлен визуально в произвольном виде или в многоуровневом параллельном виде, что означает что все вершины, располагаемые на одном уровне (ярусе) могут быть выполнены параллельно в рамках работы одного этапа алгоритма. Не следует путать такой граф с графом управления программы и тем более с её блок-схемой. Информационные графы алгоритмов активно используются при исследованиях скрытого параллелизма в алгоритмах, записанных на  последовательных языках программирования. \cite{Voevodin_book}

\subsection{Чем полезна система}

Автоматизированная система визуализации призвана обеспечить визуальное представление внутренней структуры алгоритма, облегчить процесс детального анализа алгоритма и освободить от необходимости вручную выполнять визуализацию алгоритма, облегчая исследователям возможность сосредоточиться на анализе алгоритма. Для достижения этих целей интерактивная 3D модель графа алгоритма содержит вспомогательную информацию, обеспечивающую максимальную наглядность принадлежности дуг отдельным вершинам и общей логической структуры алгоритма. Это позволяет работать с системой визуализации и тем, кто не имеет опыта в области применения интересующего алгоритма. Дополнительно система устанавливает четкую взаимосвязь со схемой работы реализации данного алгоритма и с исходным кодом реализации.

\subsection{Первичные данные}

Система визуализации, описываемая в данной работе является одной из составных частей общего проекта реализующего интерактивную трехмерную систему визуализации графов алгоритмов AlgoView. Данные, получаемые от первой части системы, выходящей за рамки данной работы являются входными данными для системы визуализации.
