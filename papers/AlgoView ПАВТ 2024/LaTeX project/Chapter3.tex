\section{Research methods and build a solution}

\subsection{Overview of existing solutions for processing XML files}

\textbf{SAX and DOM parsers.}

There are two main approaches to parsing XML files - DOM and SAX API (Application Programming Interface).

DOM analyzers process an XML file by first loading the data of this file into the program in the form of a DOM (Document Object Model) tree \cite{m6}. It is a representation of an HTML document as a tree of tags.

SAX (Simple API for XML) analyzers, in turn, carry out stream-based event processing of XML documents without loading data into the internal structures of the program \cite{m7}.

Libraries that offer a DOM API were considered as a target option, since thread processing is not required for this work.

\textbf{Libraries that provide DOM APIs for processing XML files.} The most used libraries for the C++ language that provide the DOM API are RapidXML, PugiXML, TinyXML \cite{m9}.

RapidXML is primarily focused on reducing execution time and is a stable and fast parser that runs at speeds approaching the speed of the strlen function (a function that calculates the length in bytes) running on the same data \cite{m10}.

PugiXML is a library that provides a highly traversable/modifiable DOM programming interface, an extremely fast parser, and the ability to access data along a given path \cite{m11}.

TinyXML is a minimal version for parsing XML file data and is not as fast as the previous two. This library was created for reasons of ease of use and learning, and is well suited if high execution speed is not a priority factor when choosing tools for work \cite{m12}.

\textbf{Data binding compiler.} In addition to SAX and DOM analyzers, there is a way to process it using the XSD Schema data binding compiler. This method was proposed by CodeSynthesis as an alternative solution for processing XML files with some advantages over traditional approaches \cite{m14}. Given an XML instance specification (XML Schema), it generates C++ classes representing the given vocabulary, as well as XML parsing and serialization code. Compared to APIs such as DOM and SAX, XML data binding allows the user to access data in XML documents using a generated domain vocabulary rather than generic elements, attributes, and text.

\textbf{Additional solutions for processing XML files.} In addition to the methods of working directly with XML files, it is possible to perform the same actions (access/analysis/processing) with data, only by first converting the XML format to JSON (JavaScript Object Notation) - a standard text format for data exchange. In this case, libraries for working with the JSON format are already used to process the data, the most used of which is RapidJSON.

The method of presenting data when working with the DOM programming interface turned out to be important within the framework of this work. RapidJSON works with the Value entity, which can be of two types - Object and Array. From the point of view of the source XML file, all its tags are objects and all tags of the same name are combined into arrays. Thus, the tree structure in which data is stored in the program in the RapidJSON library differs from the structures of libraries that work directly with the XML format \cite{m15}. And then bypassing this tree structure, and in particular the transition between tags of different names located at the same depth, is different and seems more convenient and safer from the point of view of memory access.

\subsection{Overview of libraries for calculating the values of mathematical expressions}

As part of the operation of the visualization system, there is a need to calculate and evaluate mathematical expressions that describe the information structure of the algorithm. For these purposes, in the C++ language there is a set of third-party open source libraries implemented into projects. The most used and suitable for this work are ExprTk, muParser, METL (Math Expression Toolkit Library).

\textbf{ExprTk library.} ExprTk is the most complete and used library for parsing and evaluating the mathematical expressions above. The parsing engine supports many forms of functional and logical processing semantics and is easily extensible. The ExprTk library allows you to work with scalar, string and vector data types with numeric types Float, Double and MPFR (multiple-precision floating-point) and supports processing various mathematical operations and functions \cite{m17}. ExprTk works on the principle of parsing an expression in an AST tree, which allows you to evaluate its value. According to performance tests, the library is considered the fastest solution for computing standard operations with scalar values.

\textbf{MuParser library series.} MuParser is a series of extensible, high-perfor-mance libraries that provide the user with different capabilities depending on his needs \cite{m19}. To achieve greater speed of working with scalar values with restrictions on the number and type of parameters used and the accuracy of calculations, the muparserSSE library is used. To overcome these limitations with increasing operating time, muparser is used. To work with vector and string data types and complex numbers, the muparserX library version is applicable.

\textbf{METL library.} Math Expression Toolkit Library is a small library for the C++14 language standard for parsing mathematical expressions. It is designed to be flexible yet effective at the same time. Flexibility means that expressions can use all types of variables with reasonable behavior (useful for working with vectors and matrices, for example) and yet adding and editing operators and functions is very easy.

\textbf{Comparison of libraries.} The libraries were compared using performance tests designed to test the correctness of calculations and the speed of operation of libraries for parsing mathematical expressions written in C++ and working on the POEM (Parse Once Evaluate Many times) principle \cite{m22}. Based on the results of calculation time measurements, the leaders among libraries for parsing mathematical expressions are muparserSSE and ExprTkFloat, working with the Float data type, and taking into account the accuracy of calculations, ExprTk, working with the Double data type.

\subsection{Research on 3D visualization methods} 3D visualization is an important aspect of user experience in software development. Currently, there is a lot of research being done in the field of graph visualizations \cite{m23}\cite{m24}\cite{m25}\cite{m26}\cite{m27}. However, analyzing the performance of algorithms using information graphs is not a very popular area in science and does not contain established methods. Understanding how to implement a project requires further consideration of the popular 3D imaging technologies available today. During the study of available 3D visualization methods, about 10 different solutions were considered: OpenGL \cite{m28}, WebGL \cite{m29}\cite{m30}\cite{m31}, Three.js \cite{m32}\cite{m33}, Babylon.js \cite{m34}, VTK (Visualization Toolkit) \cite{m35}, OpenSceneGraph \cite{m36}, DirectX \cite{m37}, Unity \cite{m38}, Blender \cite{m39}, Maya \cite{m40}.

After a comparative analysis of visualization tools, it was concluded that the simplest and most convenient methods for visualizing graphs are implemented in the Three.js, Babylon.js frameworks, as well as on the Unity platform. The final choice was made in favor of the Three.js framework due to its lightweight and cross-browser compatibility. The OrbitControls module was used to control the scene. This module is the most common and supported solution for implementing control using a computer mouse and keyboard \cite{m43}. To create a user interface and control menu with capabilities for additional analysis of algorithm graphs, the lightweight dat.GUI module was used \cite{m44}. The size of the library with all additional user interface modules does not exceed 1 megabyte.

\subsection{Method of bending edges in three-dimensional space}

\textbf{Indicator of the need to bend edges.} The main signal for the need to bend the edge occurs when the vertex is crossed by a straight edge, since in this case it is often unclear where the edge comes from and where it goes. The intersection of different edges or the passage of an edge through the shell of a 3D vertex object on average does not create problems with the visual perception of the information graph.

\textbf{The mathematical meaning of edge bending.}  Let the three-dimensional Cartesian coordinate system $\Phi$ with basis vectors $\overrightarrow{i}, \overrightarrow{j}, \overrightarrow{k}$ have 2 vertices: $A(x_1; y_1; z_1)$ and $B(x_2; y_2; z_2)$. Let us denote the length of the vector $\overrightarrow{AB}$ as $len = \sqrt{
(x_2 - x_1)^2 +
(y_2 - y_1)^2 +
(z_2 - z_1)^2 }$. Let there be a segment between vertices $A$ and $B$ that needs to be geometrically bent. By bending the edge between the vertices we can imagine that the vector $\overrightarrow{AB}$ is one of the basis vectors of another three-dimensional Cartesian coordinate system $\Psi$, in which we need to describe the equation of the curve emanating from the point $A(0; 0; 0)$, that is, the origin of coordinates, and arriving at the point $B\left(len; 0; 0 \right)$, distant from the origin of coordinates at a distance of $len$. After describing the equation of the curve, we can go to the original basis, which positions the curve in space so that it corresponds to the vertices  $A$ and $B$ in the original coordinate system $\Phi$.

\textbf{Geometric method for finding the basis.} The vector $\overrightarrow{AB}$ in the three-dimensional coordinate system $\Phi$ corresponds to the segment $AB$ and is one of the basis vectors in the coordinate system $\Psi$. Let's calculate the first unit basis vector $\overrightarrow{n_1}$:

$$
\overrightarrow{n_1} = \frac
{\overrightarrow{AB}}
{\left|\overrightarrow{AB}\right|}
$$

To find the second unit basis vector $\overrightarrow{n_2}$ we will use geometric and trigonometric methods:

$$
b = \frac
{\left|y_{n_1}\right|}
{\tan\left(
\frac{\pi}{2}
-\sin(\left|x_{n_1}\right|)
\right)};
\quad
\gamma = \arctan\left(
\left|
\frac{z_{n_1}}{x_{n_1}}
\right|
\right);
$$

$$
\overrightarrow{n_2} = \left\{
-b\cos(\gamma)
\frac{x_{n_1}}
{\left|x_{n_1}\right|};
\quad
y_{n_1};
\quad
-b\sin(\gamma)
\frac{z_{n_1}}
{\left|z_{n_1}\right|}
\right\}
$$

To find the third unit basis vector $\overrightarrow{n_3}$ we use the vector product:

$$
\overrightarrow{n_3} = \overrightarrow{n_1} \times \overrightarrow{n_2}
$$

The unit vectors $\{\overrightarrow{n_1}, \overrightarrow{n_2}, \overrightarrow{n_3}\}$ form the basis of the three-dimensional Cartesian coordinate system $\Psi$.

\textbf{Method of bending edges through a parametrically defined circle.} The bending of the edge by finding the parametric equation of the curve occurs in the above-described three-dimensional Cartesian coordinate system $\Psi$. The optimal type of curved edge is the part of the circle bounded by the points $A(0; 0; 0)$, $B\left(len; 0; 0 \right)$. To obtain the desired edge, it is enough to use the standard method of finding the equation of a circle at a given center and radius.

$$
x = R\cos\theta + x_{center};
\quad
y = R\sin\theta + y_{center}
$$

The radius of the circumscribed circle $R$ has a linear dependence on the distance $len$. The center of the circle $C(x_{center}; y_{center}; 0)$ and the range of values of the parameter $\theta$ are calculated geometrically.